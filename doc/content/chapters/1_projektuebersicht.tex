\chapter{Projektübersicht}

\section{Kontext}
Das Projekt mit dem Titel \textit{Router mit embedded Board Banana Pi R1} wird als Semesterprojekt im Wintersemester 2016/2017 an der \ac{HFU} durchgeführt und stellt somit eine von mehreren Alternativen dar, das Modul  \textit{Projektarbeit 2} zu absolvieren. Das Projekt wurde von Dr. Jiri Spale ins Leben gerufen und wird intern, ohne die Kooperation mit einem Unternehmen durchgeführt. Die Anzahl der studentischen Projektteilnehmer beträgt vier.

\section{Ziel}
Das primäre Ziel des Projektes ist, ein Router-System zu erstellen, das auf dem Banana Pi R1 laufend folgende Funktionen realisieren soll:
\begin{itemize}
	\item Die Überwachung des Internetverkehrs, getrennt auf den vier einzelnen Ethernet-Ports des Banana Pi R1.
	\item Die Archivierung der aufgezeichneten Daten.
	\item Eine \ac{VLAN}-Konfiguration, die jeweils 2 Ethernet-Ports gruppiert. Die mit einem \ac{VLAN} verbundenen Geräte sollen keinen Zugriff auf was jeweils andere \ac{VLAN} haben.
	\item Die Konfiguration des Internet-Zugriffs für die einzelnen \ac{VLAN}s.
\end{itemize}
Diese Funktionen sollen jeweils auf zwei unterschiedlichen Betriebssystemen umgesetzt werden. Außerdem gibt es weitere optionale Ziele, die nach der Realisierung des oben genannten Primärziels in Angriff genommen werden können:
\begin{itemize}
	\item Eine WLAN-Benutzerverwaltung, die registrierten Nutzern für einen begrenzten Zeitraum WLAN-Zugang ermöglicht (Vouchersystem).
	\item Die Nutzung der GPIO-Pins zur Haus-Automatisierung (\ac{IoT}).
	\item Eine Display-Statusanzeige, die den aktuellen Systemstatus anzeigt.
\end{itemize}

\section{Teilnehmer und Rollenverteilung}
Die vier studentischen Projektteilnehmer sind \docJakoby, \docKlemm, \docMeyer~und \docMichalowski. Den Projektteilnehmern ist jeweils eine Hauptrolle zugeteilt:
\begin{table}[h]
\caption{Hauptrollen}
\begin{tabular}{l|l}
\textbf{Name} & \textbf{Hauptrolle} \\
\hline
Lasse Meyer & Projektleiter \\
Stefan Jakoby & Entwickler Bananian \\
Kevin Klemm & Entwickler Netzverkehraufzeichnung und -archivierung \\
Bartosz Michalowski & Entwickler OpenWRT \\
\end{tabular}
\label{table:rollen}
\end{table}
\\[1ex]
Andere, zumeist kleinere Aufgaben, die während des Projektes aufkommen, werden dynamisch an die Projektteilnehmer verteilt.
